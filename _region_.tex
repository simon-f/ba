\message{ !name(Test.tex)}\documentclass[paper=a4, fontsize=11pt]{scrartcl} % A4 paper and 11pt font size
\usepackage[utf8]{inputenc}
% \usepackage[german]{babel} 

% for bibtex
\usepackage{cite}

\usepackage{amsmath,amsfonts,amsthm} % Math packages
\usepackage{amssymb}

\usepackage{graphicx}
\usepackage{svg}
\usepackage{import}

\usepackage{url}
\usepackage{sectsty} % Allows customizing section commands


\usepackage{titlesec}
\titleformat{\section}[block]{\Large\bfseries\filcenter}{}{1em}{}



\usepackage{tikz}
\usepackage{fancyhdr} % Custom headers and footers
\pagestyle{fancyplain} % Makes all pages in the document conform to the custom headers and footers
\fancyhead{} % No page header - if you want one, create it in the same way as the footers below
\fancyfoot[L]{} % Empty left footer
\fancyfoot[C]{} % Empty center footer
\fancyfoot[R]{\thepage} % Page numbering for right footer
\renewcommand{\headrulewidth}{0pt} % Remove header underlines
\renewcommand{\footrulewidth}{0pt} % Remove footer underlines


\setlength{\headheight}{13.6pt} % Customize the height of the header
\setlength{\parskip}{0.5em}
\setlength{\parindent}{0em}

\numberwithin{equation}{section} % Number equations within sections (i.e. 1.1, 1.2, 2.1, 2.2 instead of 1, 2, 3, 4)
\numberwithin{figure}{section} % Number figures within sections (i.e. 1.1, 1.2, 2.1, 2.2 instead of 1, 2, 3, 4)
\numberwithin{table}{section} % Number tables within sections (i.e. 1.1, 1.2, 2.1, 2.2 instead of 1, 2, 3, 4)


\title{Relations between spacial objects} 
\author{Simon Fromme}

\date{\normalsize zuletzt aktualisiert am \today}
 
\begin{document}

\message{ !name(Test.tex) !offset(78) }


\subsection{previous work}
The most simple yet common approach is to use an objects center of gravity (or some other distinguished point) and to define directional relations between spatially extended objects as 0D-0D-relationships between those points. However, this sometimes yields counter intuitive results as seen in Figure \ref{fig:center_gravity} where the center of gravity of $B$ is \emph{above} the center of gravity of $A$ although a person would probably classify the relationship the other way around. Thus using this simple approach might be easy to implement but lacks significance in complex scenarios.

\begin{figure}
  \centering
%  \includegraphics[width=3cm]
  \def\svgwidth{8em}
  \import{img/}{center_gravity.eps_tex}
  \caption{counter intuitive result when defining directional relations via the centers of gravity. The relation ``$B$ is \emph{above} $A$'' would hold in this case although that certainly violates common opinion.  }
\label{fig:center_gravity}
\end{figure}

In a two-dimensional space Borrmann and Rank \cite{Borrmann:2009:AEI} distinguish two different approaches to define directional relations between point size objects: the \textit{cone-based} and the \textit{direction-based} model. 

In the cone-based model the space around a point gets dissected into a number of equally sized partitions (usually 4 or 8) and the direction of a target point from a reference point is determined by the partition in which the latter is located. As seen in 
\message{ !name(Test.tex) !offset(119) }

\end{document}
 